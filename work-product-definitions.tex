\newcommand{\WorkProdList}{}
\newcommand{\defineWorkProduct}[3]{%
  \expandafter\def\csname identifier#1\endcsname{#2}%
  \expandafter\def\csname name#1\endcsname{#3}%
  \listgadd{\WorkProdList}{#1}
}
\newcommand{\workProdId}[1]{\textit{\csname identifier#1\endcsname}}
\newcommand{\workProdName}[1]{\csname name#1\endcsname}
\newcommand{\workProd}[1]{\workProdId{#1}. \workProdName{#1}}
\newcommand{\prodWork}[1]{\MakeLowercase{\workProdName{#1}} (\workProdId{#1})}

\defineWorkProduct{ResourceEstimates}{ĮS}{Laiko, kainos ir žmogiškųjų išteklių sąmata}
\defineWorkProduct{FunReq}{FR}{Funkciniai reikalavimai}
\defineWorkProduct{NonFunReq}{NFR}{Nefunkciniai reikalavimai}
\defineWorkProduct{HighLevelArch}{ALSA}{Aukšto lygio sistemos architektūra}
\defineWorkProduct{Experience}{PP}{Įmonės patirtis su projektais}
\defineWorkProduct{ProjectScope}{PA}{Projekto apimtis}
\defineWorkProduct{Manual}{PNI}{Produkto naudojimo instrukcija}
\defineWorkProduct{Warranty}{GAS}{Garantinio aptarnavimo sutartis}
\defineWorkProduct{Ticket}{UK}{Užregistruota klaida}
\defineWorkProduct{UserNeeds}{VP}{Vartotojų poreikiai} % FIXME
\defineWorkProduct{Product}{PROD}{Produktas} % FIXME
\defineWorkProduct{Contract}{KS}{Sutartis su klientu} % FIXME
\defineWorkProduct{Backlog}{US}{Projekto užduočių sąrašas}
\defineWorkProduct{TechDocs}{TD}{Techninė dokumentacija} % FIXME

% SCRUM

\defineWorkProduct{SprintBacklog}{SUS}{Sprinto užduočių sąrašas}
\defineWorkProduct{SprintReviewDoc}{SPA}{Sprinto peržiūros ataskaita}
\defineWorkProduct{SotryPointRange}{PVI}{Pasakojimo vienetų intervalas}

\defineWorkProduct{ProductIncrement}{PP}{Produkto prieaugis}
\defineWorkProduct{Codebase}{PK}{Programinis kodas}
\defineWorkProduct{TechDoc}{TD}{Techninė dokumentacija}
\defineWorkProduct{DefectReport}{KL}{Klaidos}
