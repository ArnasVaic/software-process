\documentclass{article}
\usepackage[T1]{fontenc}
\usepackage[lithuanian]{babel}
\usepackage{graphicx} % Required for inserting images
\usepackage{float} % For accuratelly placing images
\usepackage[a4paper, margin=2cm]{geometry}
\usepackage[hidelinks]{hyperref}
\usepackage{pdflscape}
\usepackage{longtable}
\usepackage{xltabular}
\usepackage{tabularx}
\usepackage{enumitem}

% For striketrghough
\usepackage{soul}

% For colored highlights 
\usepackage{xcolor}

% BEGIN: FOR SVG
% \usepackage[inkscapelatex=false]{svg}
\usepackage{svg}
\usepackage{amsmath}
% END: FOR SVG

\usepackage{everypage}
\usepackage{lscape} % Ensure landscape pages are recognized
\usepackage{lipsum}

\title{
    Įmonės „PTN“ procesų gerinimas\\
    \large vertinamas pagal  "AgilityMod" modelį 
    \large versija 2.0 \\
    \large Komanda „PTN“}
\author{
    Greta Virpšaitė \\
    Rugilė Vasaitytė \\
    Domantas Keturakis \\
    Arnas Vaicekauskas \\
    \textbf{Liudas Kasperavičius (Lyderis)} 
}
\date{Spalis 2024}

\begin{document}
% Globals
\newcommand{\WorkProdIdsList}{}
\newcommand{\ProcIdsList}{}

\newcommand{\CheckUniqueWorkProd}[1]{
    \ifinlist{#1}{\WorkProdIdsList} {
    \PackageError{\WorkProdIdsList}{Work product "#1" already exists}{}
    } {
    \ifinlist{#1}{\ProcIdsList} {
        \PackageError{\ProcIdsList}{"#1" exists as a Process}{}
    } {
     \listgadd{\WorkProdIdsList}{#1}
    }
  }
}

\newcommand{\CheckUniqueProc}[1]{
    \ifinlist{#1}{\ProcIdsList} {
    \PackageError{\ProcIdsList}{Work product "#1" already exists}{}
    } {
    \ifinlist{#1}{\WorkProdIdsList} {
        \PackageError{\WorkProdIdsList}{"#1" exists as a Work product}{}
    } {
     \listgadd{\ProcIdsList}{#1}
    }
  }
}

% Work products
\newcommand{\WorkProdList}{}
\newcommand{\defineWorkProduct}[3]{%
  \expandafter\def\csname identifier#1\endcsname{#2}%
  \expandafter\def\csname name#1\endcsname{#3}%
  \CheckUniqueWorkProd{#2}
  \listgadd{\WorkProdList}{#1}
}
\newcommand{\workProdId}[1]{\textit{\csname identifier#1\endcsname}}
\newcommand{\workProdName}[1]{\csname name#1\endcsname}
\newcommand{\workProd}[1]{\workProdId{#1}. \workProdName{#1}}
\newcommand{\prodWork}[1]{\MakeLowercase{\workProdName{#1}} (\workProdId{#1})}

\newcommand{\describeWorkProd}[2]{
    \expandafter\def\csname desc#1\endcsname{#2}
}
\newcommand{\printRow}[1]{
        \workProdId{#1} &
        \workProdName{#1} &
        \csname desc#1\endcsname \\ \hline
}
\newcommand{\workProdDescriptions}{
    \forlistloop{\printRow}{\WorkProdList}
}

% Processes
\newcommand{\defineProcess}[3]{%
  \expandafter\def\csname procId#1\endcsname{#2}%
  \expandafter\def\csname procName#1\endcsname{#3}%
  \CheckUniqueProc{#2}
  \listgadd{\ProcList}{#1}
}
\newcommand{\processId}[1]{\textit{\csname procId#1\endcsname}}
\newcommand{\processName}[1]{\csname procName#1\endcsname}
\newcommand{\process}[1]{\processId{#1}. \processName{#1}}


\maketitle

\newpage
\tableofcontents

\newpage

\section{Pasiruošimas vertinimui}

\subsection{Vertinimo tikslas}

Procesų gerinimas.

\subsection{Vertinimo apimties apibrėžimas}

\subsubsection{Organizacinė apimtis}
Šiame dokumente modeliuojama įmonės "PTN" departamento „Produktų vystymo“ veikla siekiant pagerinti apibrėžtus procesus (žiūrėti \textbf{1.2.3}  dokumento punktą).

\subsubsection{Aukščiausias vertinamas gebėjimo lygis}

Maksimalus vertinimas, kurį gali pasiekti procesas yra \textbf{trečias}

\subsubsection{Vertinami procesai}

Visi įmonės apibrėžti procesai (Žiūrėti pirmą įmonės apibrėžtą dokumentą) vertinami pagal "AgilityMod" modelį

%% Galėsiu padaryti (ARNAS)

\subsection{Duomenų surinkimas}

Duomenis yra/bus? renkami iš deperatemnto "PTN" procesų aprašo dokumento (TODO: čia reikia v1.3 failo pavadinimo/link'o??? i.e. "PTN procesų aprašas v1\_3.pdf")

\section{Vertinimas}

\subsection{Procesų veritinimas}

\url{https://docs.google.com/spreadsheets/d/1unX_xcZLEGHqQOMCuBBXYvVhYjChxpnq/edit?usp=share_link&ouid=113452949406463366361&rtpof=true&sd=true}

\subsection{Vertinimo rezultatai}

\subsubsection{Lentelė}
\subsubsection{"Oficialus" Gebėjimo profilis}
\subsubsection{Gebėjimo profilis gerinimui}


\section{Gerinimas}

\subsection{Tikslinis gebėjimo profilis}

\subsection{Gerinimo veiksmų planas}

\end{document}
