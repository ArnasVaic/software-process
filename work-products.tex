
\describeWorkProd{ResourceEstimates}{
Dokumentas, kuriame surašyta informacija apie atsakomybę nešančius žmones, kurie prisideda prie šio projekto vykdymo, projekto kaina bei pabaigos terminas. 
}

\describeWorkProd{ProjectScope}{
Dokumentas, apibūdinantis ką yra planuojama sukurti projekto metu, kokios numatytos sistemų funkcijos ir kokios funkcijos yra už kuriamų sistemų ribų.
}

\describeWorkProd{Manual}{Skirta sistemos naudotojams. Čia aprašomi visi panaudos atvejai, visos produkto funkcijos bei kaip jomis naudotis. PTN užtikrina teisingą produkto veikimą, jei laikomasi šio dokumento, priešingu atveju -- PTN nėra atsakinga už galimus produkto sutrikimus.
}

\describeWorkProd{Warranty}{Šis dokumentas pasirašomas perduodant klientui užbaigtą produktą. Čia numatomos sąlygos, kuriomis kliento pastebėtos produkto klaidos bus ištaisomos PTN be papildomo mokesčio per tam tikrą (taip pat šiame dokumente) numatytą laiką. Ši sutartis turi numatytą galiojimo laikotarpį.
}

\describeWorkProd{Ticket}{Tai dokumentas, užregistruotas užduočių sekimo platformoje (Jira), kuriame privalo būti ši informacija:
\begin{itemize}
    \item Registravimo data ir laikas
    \item Autorius (PTN darbuotojas arba kliento atstovas)
    \item Detalus klaidos aprašymas
    \item Kuo įmanoma detalesnis situacijos, kurioje įvyksta klaida, aprašymas
    \item Produkto versija, kurioje pastebėta klaida
\end{itemize}
Šio dokumento statusas atspindi klaidos taisymo proceso (\processId{BugFix}) stadiją:
\begin{itemize}
    \item OPEN -- klaida užregistruota
    \item IN REVIEW -- atliekama pirminė analizė
    \item REJECTED -- klaida (jei tai klaida) nebus taisoma (pridedama priežastis)
    \item IN PROGRESS -- atliekama \textit{Root Cause Analysis} ir ruošiama nauja produkto versija
    \item DONE -- nauja produkto versija išleista ir perduota klientui
\end{itemize}
}

\describeWorkProd{Backlog}{
Užduočių sąrašą sudaro bent viena užduotis. Užduotys gali būti kelių tipų:

\begin{itemize}

    \item Panaudos atvejis - tai stambi užduotis, kuri gali būti skaidoma į mažesnes užduotis, yra suformuluota iš sistemos vartotojo perspektyvos ir apibūdina sistemos funkcionalumą.

    \item Bendro pobūdžio užduotis - tai užduotis, kuri negali būti apibūdinta iš vartotojo perspektyvos, tačiau aprašo būtiną darbą sistemos veikimui užtikrinti.

\end{itemize}

Visi išvardinti užduočių tipai gali turėti vaikines, nedalomas užduotis. Taip pat, kiekviena užduotis turi tam tikrus atributus; ne visi yra iš karto priskiriami užduotims jas sukūrus, visi atributai gali keistis per laiką jei atsirastų toks poreikis. Užduočių atributų sąrašas:

\begin{itemize}
    \item Pavadinimas - trumpas pavadinimas nusakantis užduoties kontekstą 
    \item Aprašas - išsamus tekstas aprašantis užduotį
    \item Statusas - nusako kokioje stadijoje yra užduotis. Gali turėti tik viena iš šių reikšmių:
    \begin{itemize}
        \item OPEN - užduotis nepradėta
        \item IN PROGRESS - užduotis yra daroma
        \item IN REVIEW - užduotis padaryta ir reikalauja bent vieno komandos nario peržiūros
        \item TESTING - užduotis yra perduota testuotojams
        \item DONE - užduotis įvykdyta
    \end{itemize}
    \item Priėmimo kriterijai - sąlygos, kurios turi būti tenkinamos norint keisti užduoties statusą į DONE
    \item Pasakojimo vienetai - skaliarinis įvertis, kuris nusako reliatyvų užduoties sudėtingumą.
    \item Atsakingas asmuo - šiuo metu užduotį atliekantis arba testuojantis asmuo.
    \item Prioritetas - užduoties svarba. Vertinama reliatyviai, t. y. kuo užduočių sąraše užduotis yra aukščiau, tuo užduotis turi būti greičiau atlikta.
    \item Kūrimo valandos - užduočiai įgyvendinti skiriamos valandos.
    \item Testavimo valandos - užduočiai testuoti skiriamos valandos.
\end{itemize}

}


% -------------- END OF DESCRIPTIONS-------------------------
\section{Darbo produktų sąrašas}

\begin{longtable}{|c|p{0.2\textwidth}|p{0.75\textwidth}|}
    \hline
    \textbf{Id} & \textbf{Pavadinimas} & \textbf{Aprašymas} \\ \hline
    \workProdDescriptions
\end{longtable}