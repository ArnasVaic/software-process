\newcommand{\describeWorkProd}[2]{
    \expandafter\def\csname desc#1\endcsname{#2}
}
\newcommand{\printRow}[1]{
        \workProdId{#1} &
        \workProdName{#1} &
        \csname desc#1\endcsname \\ \hline
}
\newcommand{\workProdDescriptions}{
    \forlistloop{\printRow}{\WorkProdList}
}

\describeWorkProd{Manual}{Skirta sistemos naudotojams. Čia aprašomi visi panaudos atvejai, visos produkto funkcijos bei kaip jomis naudotis. PTN užtikrina teisingą produkto veikimą, jei laikomasi šio dokumento, priešingu atveju -- PTN nėra atsakinga už galimus produkto sutrikimus.
}

\describeWorkProd{Warranty}{Šis dokumentas pasirašomas perduodant klientui užbaigtą produktą. Čia numatomos sąlygos, kuriomis kliento pastebėtos produkto klaidos bus ištaisomos PTN be papildomo mokesčio per tam tikrą (taip pat šiame dokumente) numatytą laiką. Ši sutartis turi numatytą galiojimo laikotarpį.
}

\describeWorkProd{Ticket}{Tai dokumentas, užregistruotas užduočių sekimo platformoje (Jira), kuriame privalo būti ši informacija:
\begin{itemize}
    \item Registravimo data ir laikas
    \item Autorius (PTN darbuotojas arba kliento atstovas)
    \item Detalus klaidos aprašymas
    \item Kuo įmanoma detalesnis situacijos, kurioje įvyksta klaida, aprašymas
    \item Produkto versija, kurioje pastebėta klaida
\end{itemize}
Šio dokumento statusas atspindi klaidos taisymo proceso (BugFix) stadiją:
\begin{itemize}
    \item OPEN -- klaida užregistruota
    \item IN REVIEW -- atliekama pirminė analizė
    \item REJECTED -- klaida (jei tai klaida) nebus taisoma (pridedama priežastis)
    \item IN PROGRESS -- atliekama \textit{Root Cause Analysis} ir ruošiama nauja produkto versija
    \item RESOLVED -- nauja produkto versija išleista ir perduota klientui
\end{itemize}
}


% -------------- END OF DESCRIPTIONS-------------------------
\section{Darbo produktų sąrašas}

\begin{longtable}{|c|c|p{0.5\textwidth}|}
    \hline
    \textbf{Identifikatorius} & \textbf{Pavadinimas} & \textbf{Aprašymas} \\ \hline
    \workProdDescriptions
\end{longtable}