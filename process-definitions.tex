\newcommand{\ProcList}{}
\newcommand{\defineProcess}[3]{%
  \expandafter\def\csname procId#1\endcsname{#2}%
  \expandafter\def\csname procName#1\endcsname{#3}%
  \listgadd{\ProcList}{#1}
}
\newcommand{\processId}[1]{\textit{\csname procId#1\endcsname}}
\newcommand{\processName}[1]{\csname procName#1\endcsname}
\newcommand{\process}[1]{\processId{#1}. \processName{#1}}

% ---------------- Define Processes

\defineProcess{EngageClient}{KĮ}{Kliento įtraukimas}
\defineProcess{RA}{RA}{Reikalavimų analizė}
\defineProcess{DraftBacklog}{UR}{Užduočių sąrašo rengimas}

\defineProcess{CreateManual}{ND}{Naudojimo dokumentacija}
\defineProcess{CloseProject}{PU}{Projekto užbaigimas}
\defineProcess{BugFix}{KT}{Klaidos taisymas}


% -----------------------Process Table Environment
\newenvironment{processTable}[1]
{
\newcommand{\tikslas}[1]{\gdef\purpose{##1}}
\newcommand{\inputs}[1]{\gdef\usedWorkProducts{##1}}
\newcommand{\outputs}[1]{\gdef\producedWorkProducts{##1}}
\newcommand{\veiklos}[1]{\gdef\activities{##1}}
\def\purpose{}
\def\usedWorkProducts{}
\def\producedWorkProducts{}
\def\activities{}
\def\processPK{#1}
}
{
\begin{table}[h]
\label{processTable:\processPK}
\begin{tabular}{p{0.175\textwidth}|p{0.725\textwidth}}

\textbf{\processId{\processPK}} & \textbf{\processName{\processPK}} \\ \hline
Tikslas & \purpose 
\\ \hline
Naudojami darbo produktai & \begin{itemize} \usedWorkProducts \end{itemize}
\\ \hline
Sukuriami darbo produktai  & \begin{itemize} \producedWorkProducts \end{itemize} 
\\ \hline
Veiklos & \begin{enumerate} \activities \end{enumerate}
\end{tabular}
\end{table}
}

% \begin{processTable}{PrimaryKey}
%     \tikslas{}
%     \inputs{
%        \item \workProd{Kazkas}
%     }
%     \outputs{
%        \item \workProd{Kazkas}
%     }
%     \veiklos{
%         \item Kazkokia veikla
%     }
% \end{processTable}