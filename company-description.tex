\section{Įmonės aprašymas}

\subsection{Įmonės pavadinimas}
"PTN"

% \subsection{"PTN" departments}

% \begin{enumerate}
%     \item  IT division
% \end{enumerate}

\subsection{Įmonės aprašymas}

% \textbf{NOTE:} Reikia aprašyti TIK IT depertamentą \newline

"PTN" yra projektinė įmonė. "Produktų vystymo" departamentas yra "PTN" IT departamentas, kuris užsiima e.komercijos sistemų kūrimu klientams. "Produktų vystymo" departamentas įdarbinę apie 30 darbuotojų. 

\subsection{Organizacinė struktūra}
Šiame dokumente mes modeliuojame tik IT departamentą pavadinimu "Produktų vystymo", kuris prisiima projektus susijusius su e.komercija.
\begin{table}[h!]
\centering
\begin{tabular}{p{0.1\textwidth}|p{0.9\textwidth}}
\hline
\textbf{Rolės} & \textbf{Atsakomybės} \\ \hline


% product developement & 

Projektų vadovas & \st{Manages and supervises the project, including setting project goals, timelines, and budgets. The project manager ensures that the development team meets deadlines and that the project is delivered successfully within the agreed-upon scope. They also act as the main point of communication between the client and the development team.}

Valdo ir prižiūri projektą, įskaitant projekto tikslų, terminų ir biudžetų nustatymą. Projekto vadovas užtikrina, kad programinės įrangos kūrimo (software development) komanda laikytųsi terminų ir kad projektas būtų sėkmingai įgyvendintas sutarta apimtimi. Jie taip pat tarpininkauja kliento ir programinės įrangos kūrimo (software development) komandos bendravime.

\\ \hline
% Deployment & Integrate developed systems with existing client platforms, for example: inventory or payment gateways. Maintain the internal IT infrastructure of the client companies. \\ \hline
Programinės įrangos kūrėjas (Software developers) & \st{ Responsible for developing the custom e-commerce software in line with the project's specifications. They ensure the functionality of the system through testing and work closely with the project manager to make sure client requirements are met. Developers also address all issues or bugs that arise during the development process. }

Atsakingas už pritaikytos e.komercijos programinės įrangos kūrimą pagal projekto specifikacijas. Testuotojai užtikrina sistemos funkcionalumą testuodami ir glaudžiai bendradarbiaudami su projekto vadovu, siekiant užtikrinti, kad būtų patenkinti kliento reikalavimai. Programinės įrangos kūrėjai (Software developers) taip pat sprendžia visas problemas ar sistemos klaidas, kurios kyla kūrimo proceso metu.
\\ \hline

Testuotojas & \st{Responsible for evaluating the functionality and quality of software. They conduct various types of testing to identify defects before the product is delivered and work closely with developers to validate fixes to their reported defects. }


Atsakingas už programinės įrangos funkcionalumo ir kokybės įvertinimą. Jie atlieka įvairių tipų testavimą, kad nustatytų defektus prieš pristatant produktą ir glaudžiai bendradarbiauja su programinės įrangos kūrėjais (Software developers), kad patvirtintų praneštų defektų pataisas.
\\ \hline

Architektas & \st{Designs the overall structure of the e-commerce software. The architect collaborates with developers to guide implementation and resolve complex technical challenges. }
Sukuria bendrą e.komercijos programinės įrangos struktūrą. Architektas bendradarbiauja su programininės
įrangos kūrėjais (software developers), siekdamas juos vesti įgyvendinimo procese ir padeda išspręsti sudėtingus techninius iššūkius.

\\ \hline
Analistas &  \st{Gather and analyse client requirements to ensure the project meets business needs.}

Surinka ir analizuoja klientų poreikius, kad įsitikintų, jog projektas atitinka verslo poreikius.
\\ \hline
% Projects & Manage and supervise project progress, develop project plans, and ensure the successful delivery of these systems within the agreed timelines and budgets. \\ \hline
% System Maintenance & Provide ongoing maintenance for live e-commerce platforms. Communicate with clients regarding system performance and to resolve any incidents or issues that arise. \\ \hline

% Paminėti, kad garantiniai atvejai

\end{tabular}
%\caption{Organizational Structure}
%\label{table:organizational_structure}
\end{table}


% \subsection{IT division}

% The IT department employs around 30 people. It carries out projects for private and public organisations and develops and maintains web applications in various fields.

% Short department description

% Project-based organisational structure

% E-commerce

% Agile/SCRUM

% Develops products and transfers them to the customer

% \subsubsection{Additional information}

% \paragraph{Organisational structure}
% \paragraph{Technologies used}
% \paragraph{Methodologies}
