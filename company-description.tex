\section{Įmonės aprašymas}

\subsection{Įmonės pavadinimas}
"PTN"

% \subsection{"PTN" departments}

% \begin{enumerate}
%     \item  IT division
% \end{enumerate}

\subsection{Įmonės aprašymas}
"PTN" is a project-based company. "PTN-1" is a team part of "PTN" that is responsible for delivering customized software solutions for e-commerce clients. "PTN-1" project develops customised software such as managing inventory, processing online payments, handling customer relationship management (CRM), and automating order fulfillment processes. Each client can have multiple custom-built systems designed to enhance their e-commerce operations. "PTN-1" employs between 20 and 50 workers, although this number varies depending on the scope of the project.


\subsection{Organizacinė struktūra}
\begin{table}[h!]
\centering
\begin{tabular}{|l|p{0.8\textwidth}|}
\hline
\textbf{Rolės} & \textbf{Atsakomybės} \\ \hline
% product developement & 

Project manager & Manages and supervises the project, including setting project goals, timelines, and budgets. The project manager ensures that the development team meets deadlines and that the project is delivered successfully within the agreed-upon scope. They also act as the main point of communication between the client and the development team. \\ \hline
% Deployment & Integrate developed systems with existing client platforms, for example: inventory or payment gateways. Maintain the internal IT infrastructure of the client companies. \\ \hline
Software developer & Responsible for developing the custom e-commerce software in line with the project's specifications. They ensure the functionality of the system through testing and work closely with the project manager to make sure client requirements are met. Developers also address all issues or bugs that arise during the development process. \\ \hline
Tester & Responsible for evaluating the functionality and quality of software. They conduct various types of testing to identify defects before the product is delivered and work closely with developers to validate fixes to their reported defects. \\ \hline
Architect & Designs the overall structure of the e-commerce software. The architect collaborates with developers to guide implementation and resolve complex technical challenges. \\ \hline
Analyst &  Gather and analyse client requirements to ensure the project meets business needs.\\ \hline
% Projects & Manage and supervise project progress, develop project plans, and ensure the successful delivery of these systems within the agreed timelines and budgets. \\ \hline
% System Maintenance & Provide ongoing maintenance for live e-commerce platforms. Communicate with clients regarding system performance and to resolve any incidents or issues that arise. \\ \hline

% Paminėti, kad garantiniai atvejai

\end{tabular}
%\caption{Organizational Structure}
%\label{table:organizational_structure}
\end{table}


% \subsection{IT division}

% The IT department employs around 30 people. It carries out projects for private and public organisations and develops and maintains web applications in various fields.

% Short department description

% Project-based organisational structure

% E-commerce

% Agile/SCRUM

% Develops products and transfers them to the customer

% \subsubsection{Additional information}

% \paragraph{Organisational structure}
% \paragraph{Technologies used}
% \paragraph{Methodologies}
